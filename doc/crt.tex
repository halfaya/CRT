\documentclass[12pt]{article}
\usepackage[utf8]{inputenc}
\usepackage[english]{babel}
\usepackage[]{amsmath}
\usepackage[]{amsthm}
\usepackage[]{amssymb}
\usepackage[margin=0.5in]{geometry}
\usepackage[colorlinks=true, urlcolor=blue, linkcolor=red]{hyperref}

\newtheorem*{theorem}{Theorem}

\title{Chinese Remainder Theorem}
\author{John Leo}
\date{\today}

\begin{document}
\maketitle

\section{Instrumentation}

\begin{enumerate}
\item Xiaoluo [small gong] (30 bar phrases)
\item Paigu [drums] (7 bar phrases)
\item Yanqin (5 bar phrase)
\item Plucked Strings (3 bar phrases)
  \begin{enumerate}
  \item Zhongruan
  \item Pipa
  \end{enumerate}
\item Bowed Strings (2 bar phrases)
  \begin{enumerate}
  \item Erhu
  \item Zhonghu
  \end{enumerate}
\end{enumerate}

\section{Musical Structure}

\begin{itemize}
\item There are $2\cdot 3\cdot 5\cdot 7 = 210$ bars of music.

\item Tempo is 120 bpm. 4/4 time, so 840 beats and exactly 7 minutes of music.
Divide into 7 one-minute sections of 30 bars of music.

\item The key signature is G major. However they key could also be
  viewed as D major, as C/C\# is avoided.

\item All instruments play only scale tones.

\begin{itemize}
\item Yangqin was originally to play the pentatonic scale omitting 4 and 7 (G, A, B, D, E).
  \\
  (Major Pentatonic: 
  \href{https://ianring.com/musictheory/scales/661}{https://ianring.com/musictheory/scales/661})
  \\
  However it ended up playing the same hexatonic scale as the others.

\item All non-percussion instruments play a hexatonic scale omitting only 4 (G, A, B, D,
  E, F\#). \\
  (\href{https://ianring.com/musictheory/scales/2709}{https://ianring.com/musictheory/scales/2709})\\
  If the key is viewed as D major then 7 is omitted and the scale is the major diatonic hexachord\\
  (\href{https://ianring.com/musictheory/scales/693}{https://ianring.com/musictheory/scales/693})\\
\end{itemize}
\end{itemize}

\section{Chinese Remainder Theorem}

\begin{theorem}[Chinese Remainder Theorem]
  Let $I_1,\ldots,I_N$ be pairwise comaximal ideals of commutative
  ring $R$ ($I_i+I_j = R$ for all $i\ne j$). Then there is a ring
  isomorphism $R / (\cap_k I_k) \simeq \prod_k (R / I_k)$. In
  particular if $n_k \in \mathbb{Z}$ are pairwise coprime then
  $\mathbb{Z} / [(\prod_k n_k)\mathbb{Z}] \simeq \prod_k (\mathbb{Z} /
  n_k\mathbb{Z})$.
\end{theorem}

\begin{proof}[Proof Outline]
  The primary source for the proof is Serge Lang, \textit{Algebra,
    Revised Third Edition} (2002), pages 94-95.
  \begin{enumerate}
    \item Bézout’s Lemma: Coprime implies comaximal. If $(m,n) = 1$,
      then $\exists a,b \text{ s.t. } 1 = am + bn$.
    \item Surjectivity. Base case $n=2$. Given $(c,d) \in \mathbb{Z} /
      m\mathbb{Z} \times \mathbb{Z} / n\mathbb{Z}$, let $x =dam + cbn$.
    \item Induction: For all $k$, $n_k\mathbb{Z}$ is comaximal to the
      product of the other $n_i\mathbb{Z}$: $n_k\mathbb{Z} + \prod_{i
      \ne k}n_k\mathbb{Z} = \mathbb{Z}$.
    \item By the theorem with $n=2$, for all $k$ we can find $y_k$
      satisfying $y_k \equiv 1 \bmod n_k$, $y_k \equiv 0 \bmod \prod_{i\ne
        k}n_i$.
    \item Then given $(x_1,\ldots,x_N) \in \prod_k (\mathbb{Z} /
      n_k\mathbb{Z})$, $\sum_k x_ky_k$ maps to $(x_1,\ldots,x_N)$.
    \item The map $f : \mathbb{Z} \to \prod_k (\mathbb{Z} /
      n_k\mathbb{Z})$ defined by $f(x) = \prod_k (x \bmod n_k)$ is a
      (surjective) ring homomorphism.
    \item The kernel of $f$ is $(\prod_k n_k)\mathbb{Z}$.
  \end{enumerate}
\end{proof}

\section{Musical Realization}

Xiaoluo introduces each 30 bar section, counting the section
number. Paigu plays patterns changing every 7 bars so that the full
cycle is completed in 210 bars. Each section is described below.

\begin{enumerate}

\item Bézout’s Lemma
  (\href{https://leanprover-community.github.io/mathlib4_docs/Mathlib/Algebra/EuclideanDomain/Basic.html#EuclideanDomain.gcd_eq_gcd_ab}{EuclideanDomain.gcd\_eq\_gcd\_ab})
  is illustrated by the special case $m = 2, n = 3$ following
  the extended Euclidean algorithm
  (\href{https://leanprover-community.github.io/mathlib4_docs/Mathlib/Algebra/EuclideanDomain/Defs.html#EuclideanDomain.xgcd}{EuclideanDomain.xgcd}).
Erhu/Zhonghu represent
2, Pipa/Zhongruan represent 3, and Yangqin does additional work. The
sequence (following
\href{https://leanprover-community.github.io/mathlib4_docs/Mathlib/Algebra/EuclideanDomain/Defs.html#EuclideanDomain.xgcdAux}{EuclideanDomain.xgcdAux}) is:

\begin{enumerate}
\item $2 = 1\cdot 2 + 0\cdot 3$
\item $3 = 0\cdot 2 + 1\cdot 3$
\item $2 = 1\cdot 2 + 0\cdot 3$
\item $1 = (-1)\cdot 2 + 1\cdot 3$
\end{enumerate}

Each step takes 6 bars. The final 6 bars is devoted to a special case
of the surjectivity base case: $2 = (-2)\cdot 2 + 2\cdot 3$

\item Surjectivity. Base case $n=2$.
Again for $m = 2, n = 3$, five of the six cases (omitting only
  $(0,0)$) are illustrated using different themes in this order:

\begin{enumerate}
\item $2 = 1\cdot 2 + 0\cdot 3$ [$2 \leftrightarrow (0,2)]$
\item $4 = 2\cdot 2 + 0\cdot 3$ [$4 \leftrightarrow (0,1)]$
\item $3 = 0\cdot 2 + 1\cdot 3$ [$3 \leftrightarrow (1,0)]$
\item $5 = 1\cdot 2 + 1\cdot 3$ [$5 \leftrightarrow (1,2)]$
\item $7 = 2\cdot 2 + 1\cdot 3$ [$1 \leftrightarrow (1,1)]$
\end{enumerate}

\item Illustration of the beginning of induction using $2,3,5$: Each
  number is comaximal to the product of the other two.

\begin{enumerate}
\item $1 = (-1)\cdot 5 + 1\cdot 6$
\item $1 = (-3)\cdot 3 + 1\cdot 10$
\item $1 = (-7)\cdot 2 + 1\cdot 15$
\end{enumerate}

The sections are 5, 10, 15 measures long respectively. The basic idea
is that the ``lone'' instrument (or pair) plays in opposition to the
other two.

\item Inductive hypothesis with $n=2$ illustrated for five of the
  thirty cases for $m = 5, n = 6$.

\begin{enumerate}
\item $11 = 1\cdot 5 + 1\cdot 6$ [$11 \leftrightarrow (1,5)]$
\item $16 = 2\cdot 5 + 1\cdot 6$ [$16 \leftrightarrow (1,4)]$
\item $17 = 1\cdot 5 + 2\cdot 6$ [$17 \leftrightarrow (2,5)]$
\item $22 = 2\cdot 5 + 2\cdot 6$ [$22 \leftrightarrow (2,4)]$
\item $27 = 3\cdot 5 + 2\cdot 6$ [$27 \leftrightarrow (2,3)]$
\end{enumerate}

\item 5
\item 6
\item 7

\end{enumerate}

\end{document}
